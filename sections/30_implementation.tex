\section{Implementation}\label{sec:implementation}

The general flow of CBR will look like this: After a successful shot, a new case will be created or an existing one will be updated. When looking for shots to execute, the database of cases can be scanned for matching cases and a plan for a shot can be generated for every match.

% What is a Case in the context of Angry Birds?

\subsection{A case in AngryBirds}\label{subsec:impl-case}
What does a case in the context of AngryBirds consist of? There are three essential parts: A description of the situation in which this case is valid, a description of the shot and a description of the effects, that we expect this shot to have in this situation.

\paragraph{The situation}
A description of the situation is necessary so the planner can determine, whether a given scenario is sufficiently similar for the case to provide meaningful input.
For the qualitative CBR, this description consists of a every object that had been affected by the original shot and a set of relations, that provide more information about those objects and their relation to each others.


\paragraph{The shot}
For the quantitative CBR we has used the target coordinates of the original shot, which were then transformed to a matched scenario.
Because this Transformation of coordinates will not happen during the qualitative CBR process, a shot is instead represented by an object that the shot was aimed at.
A drawback of this approach is the inability to do complex shots, that do not directly target an object.
Examples of such complex shots would be rebound shots.
Additional information is added however, as along with the target object, qualitative predicates about the shot are saved. These include classification of the trajectory (high vs low shots) and where the target was (supposed to be) hit, i.e. whether the target was above, below or at the center of the target object.
The latter can only be estimated, however. Due to technical restrictions we can record where the agent was trying to hit the object, but not where the shot actually hit.


\paragraph{The Effects}
There are two main use cases for the expected effects a shot might have. The first is in the planning phase, where these might be used to determine whether this shot will improve the situation at hand, thus enabling complex planning processes for multiple shots in a row.
After executing a shot, the expected effects can be compared to the observed effects. If the effects match even though the situation was not a perfect match, this can be used to update the case in order to loosen restrictions on the situation.
If, on the other hand, the situation did match well but the observed effect was notably different from the expected effect,
% TODO: What then??

\paragraph{When should a new case be added?}
This question goes back to an entirely different, not yet solved problem: Determining how good a shot was.
This is nontrivial, as a shot might not kill a target but still improve the overall situation by removing obstacles and enabling better shots at remaining targets.
A shot that takes out a target might also not be a very good shot if it leads to other targets becoming obstructed and harder to hit.
Because classifying the quality of a shot is out of scope for this paper, we are gonna assume that shots are good iff they kill one or more targets.
When the agent has executed a shot that meets this criterion, on of two things can happen: The shot was already based on a CBR shot, then the underlying case can be updated based on the observed effects. Otherwise, the a new case is generated using this shot and its effects.

\paragraph{Determining conditions for applicability of a case}
The original idea for matching cases was to add a prolog rule for every new case, consisting of a set of required predicates and a set of optional predicates.
This rule would generate configurations of objects matching every required predicate for a given case and then check the ratio of optional predicates that hold for this configuration.
This is necessary because we don't only want perfect matches, but close matches as well which can be used to update and generalize cases.

This is therefore an instance of a more general Constraint Satisfaction Problem (Max CSP). which is known to be NP hard.

Thus efficiently implementing this part is a major challenge:
Configurations cannot be discarded as soon as the first relation fails to hold.
Even if the only required predicates are the actual objects, i.e. every object must have a counterpart in the scene but all additional information, such as materials, shape and position are optional, the number of possible permutations that have to be checked are too high to be efficiently checked.

Unfortunately, runtime performance is actually important here. While case generation and updating can easily be background tasks that are run separately from the actual agent, the detection of applicable strategies needs to happen in real time.
In order to maximize the amount of time spent performing shots, and thus potentially solving levels and increasing score, the strategy generation needs to be as fast as possible.

% TODO: add more data here as well
A possible mitigation could be including more information about the objects as required predicates instead of optionals.
The most obvious candidates for this would be shape and material.
% TODO: how much improvement in expectation?
The drawback is that cases cannot be generalized over those attributes anymore. While for some cases the material of a block might be important, especially when the goal lies in breaking a block, for other cases the material might not matter.
With this approach, this will not be detected, however, since the case will never match to other materials und thus cannot be updated.

Another way to decrease the number of possible permutations would be decreasing the number of objects associated with a case. Along with an expected decrease in the accuracy of the system, this approach would probably decrease the value for complex planning, as per the definition every object associated with the case is expected to be affected by the shot und thus should be considered during planning.
Determining which objects to omit would also be non-trivial.

When implemented in Prolog, there is an issue with its backtracking search: Consider a case with 10 objects: If, for a given configuration of 10 matched objects, none of the optional predicates match, exchanging one of the 10 matched objects cannot possible meaningfully change the result
% TODO: add more data here i guess
Instead, a randomized approach could be helpful, where the entire configuration is discarded when the percentage of matching optional predicate is below a threshold and a completely new configuration is chosen.

One possibility would be generating a fixed amount of random configurations matching the hard constraints for each case, checking the soft constraints for each configuration. This is not really helpful however. Test show, that even in situations where a perfect match is available, the best match of 10000 iterations on average is slightly over half of predicates % TODO: add data here 


\paragraph{Determining effects of a shot}
In the previous section, we identified all objects that were impacted by a shot.
To determine effects, those objects need to be

\paragraph{Relations for describing objects}
Apart from the rather obvious properties of single objects, such as their general shape("rectangle", "circle", or "poly"), their material("ice", "wood", "stone", "tnt") and their rough dimensions, a key constraint for similarity of scenes is the relative position of objects to each other.
There are different approaches with different precision.
The most precise description can be achieved using the Extended Rectangle Algebra introduced by (% TODO: cite paper here
). Because this was intended to determine stability of structures, this provides very granular information about objects that are close or touching.
A possible addition could consist of adding information whether object A is close enough to object B such that A could impact B when falling over, which would be encoded as before by ERA either way.

% How are cases generated?
% How are effects determined?
% How are predicates/relations used and determined?
% Can case be used/Which case should be used?
% "Random" shots of no case matches
% Updating cases after success/failure
% protection against trivial cases?