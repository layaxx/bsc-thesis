The main idea behind incorporating CBR in AngryBirds is to save interesting shots for future use. Therefore shots that directly target pigs are of little interest and are not saved to the database.
To increase the chance of finding unusual shots, the agent was modified to perform shots at random objects. For rectangular objects, the shots are aimed at either one, two or three fourth of the object's height instead of the center, called randomBottom, randomCenter and randomTop respectively.

Because other objects are only targeted at the center, randomCenter shots are more prevalent, as can be seen in \ref{tab:strat-distribution}, which shows considered and executed shots generated by the randomShot strategy over a 60 min run.
The demo strategy is a fallback strategy employed by the agent if the planner module fails to find any plans, either because of a runtime error or because no object can be targeted without obstacles. For this reason, it is only present in the executed shots, but not in the candidates.

\asfigure{fig:random-shot}{data/randomshot}{Visualization of shots the randomShot strategy generated on Level 7}{15}

\begin{table}[b]
    \centering
    \begin{tabular}{l|l|l}
                              & \textbf{candidates} & \textbf{executed} \\ \hline
        \textbf{demo}         & 0\%                 & 00.56\%           \\ \hline
        \textbf{randomTop}    & 15.00\%             & 11.73\%           \\ \hline
        \textbf{randomCenter} & 71.36\%             & 73.18\%           \\ \hline
        \textbf{randomBottom} & 15.00\%             & 14.53\%
    \end{tabular}
    \caption{Frequency of strategies considered and executed while building the case base for section \ref{subsec:experimental-predicates}}
    \label{tab:strat-distribution}
\end{table}