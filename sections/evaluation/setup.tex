The exact experimental setup depends on the component under test and is described in more detail in the relevant sections.

In general, a version of the agent runs on a set of levels, which is detailed below.
A script then logs relevant information during runs, such that they can be analyzed and compared afterward.
The most important metric is which levels the agent is capable of beating at all and the highest achieved score per level.
Other data includes which strategies were used for each shot and the effects of \ac{CBR} shots.
This is especially important to determine the frequency with which \acs{CBR} cases are found to be applicable and the success rate of adaptation.

Some of the evaluations are performed on a database of cases. Unless otherwise mentioned, these are built by running a modified version of the agent, which performs shots derived from the ``randomShot'' strategy derived earlier. If a shot destroys a pig without targeting it directly, the effects are analyzed and the appropriate Prolog rules are generated and saved.

All evaluations were performed on a Linux system (OpenSUSE Tumbleweed) with 16GB of RAM and a 2.6 Ghz hexacore CPU. Relevant runtime environments are Java 17.0.5 and SWI-Prolog version 8.4.1.

The level set consists of 21 levels, that have been used in the past to evaluate domino strategies. They have been selected for the evaluation set because they include some instances of similar object configurations in different levels while still being diverse.
On average, the levels consist of 40.67 entities (min: 12, max: 68, median: 36). An entity is an object than can be affected by a shot, e.g. a block of ice or a pig, but not a hill or a bird.

The level set contains on average (median) four pigs and four birds per level. 
The distribution of materials and shapes of the objects can be seen in figure \ref{fig:objects-in-levels}. 
This is relevant to the analysis of the number of possible configurations of a case with hard constraints, as discussed in section \ref{subsec:experimental-applicability}.
The structure of the levels can be seen in the Appendix, figure \ref{fig:levels}.

\asfigure{fig:objects-in-levels}{data/level_objects}{Distribution of shapes and material over the 21 Levels}{15}