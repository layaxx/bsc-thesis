\section{Introduction}\label{sec:intro}

Angry Birds is physics-based puzzle game where the player has to hit every target (``pig'') with a set number of ballistic projectiles (``birds'') in order to clear a level and achieve a score.
There are two main kinds of obstacles: blocks that can be moved or destroyed and hills that can not be impacted. Targets may be hidden behind or under any kind of obstacle.

While humans are generally able to achieve good scores on levels, this is not a trivial task for computers.
Creating an agent capable of doing this is the main goal of AIBIRDS, a yearly competition that started in 2012 \cite{Renz2015AIBIRDSTA}.

This paper describes the implementation and evaluation of a qualitative case-based reasoning strategy for such an agent, with the goal of providing alternatives to current strategies and potentially improve scoring performance.

The next section describes the background of the work. Then the approach is described, followed by details of the implementation and evaluation results.

\subsection{Previous Work}
\paragraph{Quantitive Case-Based reasoning}\label{par:quantititve-cbr}
In previous work, we devised a strategy to use case-based reasoning to allow the agent to remember good shots und use them again in similar scenarios, even though the strategy that originally produced this shot may not have a high priority.
A case in this context consisted of a set of affected objects (see \ref{subsec:impl-effects}) and the coordinates of the point at which the bird was aimed.

A case was applicable to a given scene if a translation exists, such that for every object associated with a case, the scene has an object of the same material and area at the translated coordinates. A translation consists of a shift along the x-axis, a shift along the y-axis as well as a scaling factor.

% TODO: probably already described above
%% Case-based reasoning is a strategy in artificial intelligence that aims to %% modify known solutions to problems in order to make them applicable to the %% problem at hand.
%% To do so, relevant information about the problem and the associated solution %% need to be stored and similarity between problems must be judged.
%% If two problems are similar enough, the solution can be transformed to solve %% the new problem.


\cite{QCBR} discusses a similar problem, applying a combination of \ac{CBR} and Reinforcement Learning to robotic soccer players.
The qualitative spatial representation framework they use, \ac{EOPRA} is one of the options discussed in \ref{subsec:impl-relations}.



\subsection{Goals \& Approaches}

Now the goal is to devise a qualitative solution.
Instead of absolute positions, relations between objects shall be considered when determining applicability of cases.
This could lead to an improvement in performance compared to the quantitative approach, as that came with limitations as to what objects were considered and when two scenes where estimated to be transferable. % TODO: should evaluation also be run against the quantitative approach?
For example, a few pixels could make the difference between an object being irrelevant or an actual obstacle.
But due to thresholds we used in the quantitative approach, this distinction may not occur. % TODO: add image of a level where this could be an issue
% TODO: Check wether this level could then be beaten using QualCBR

The main objective will consist of comparing subsets of quantitative relations to find those that maximize the score the agent can achieve in-game as well as finding a suitable method for quantifiying scene similarity.
