\section{Introduction}\label{sec:intro}

Angry Birds is a physics-based puzzle game where the player has to hit every target (``pig'') with a set number of ballistic projectiles (``birds'') in order to clear a level and achieve a score.
There are two main kinds of obstacles: blocks that can be moved or destroyed and hills that can not be impacted. Targets may be hidden behind or under any kind of obstacle.

While humans are generally able to achieve good scores on levels, this is not a trivial task for computers.
Creating an agent capable of doing this is the main goal of AIBIRDS, a yearly competition that started in 2012 \cite{Renz2015AIBIRDSTA}.

\ac{CBR} is a strategy in artificial intelligence in which known solutions to problems are modified in order to make them applicable to the problem at hand.
To do so, relevant information about the problem and the associated solution need to be stored and the similarity between problems must be judged.
If two problems are similar enough, the solution can be transformed to solve the new problem.

One advantage \ac{CBR} has over other AI paradigms like Deep Learning is that it requires less data and is more explainable \cite{explainable-https://doi.org/10.48550/arxiv.1710.04806}. Since the agent generates cases from its own experiences, general knowledge of the problem domain is not required \cite{CBR-issues-variations-approaches}.


\subsection{Previous Work}\label{subsec:previous}
In previous work, we devised a strategy to use case-based reasoning to allow the agent to remember good shots and use them again in similar scenarios, even though the strategy that originally produced this shot may not have a high priority.
A case in this context consisted of a set of affected objects (see \ref{subsec:impl-effects}) and the coordinates of the point at which the bird was aimed.

A case was applicable to a given scene if a translation exists, such that for every object associated with a case, the scene has an object of the same material and size at the translated coordinates. A translation consists of a shift along the x and y-axis as well as a scaling factor.

\cite{QCBR} discusses a similar problem, applying a combination of \ac{CBR} and Reinforcement Learning to robotic soccer players.
The qualitative spatial representation framework they use, \ac{EOPRA} is one of the candidates analyzed in \ref{subsec:impl-relations}.

\subsection{Goals \& Approaches}
Now the goal is to assess whether a \ac{CBR} strategy based on qualitative representations is viable for an AIBIRDS agent.
Instead of absolute positions, relations between objects shall be considered when determining the applicability of cases.
This could lead to an improvement in performance compared to the quantitative approach, as that came with limitations as to what objects were considered and when two scenes were estimated to be transferable. For example, a few pixels could make the difference between an object being irrelevant or an actual obstacle.
But due to the thresholds we used in the quantitative approach, this distinction may not occur.

There are two main questions: What qualitative relations between objects should be used such that cases can be transferred successfully and the agent can achieve a score that is as high as possible?
How can similarity between scenes be quantified, such that the best matching case can be found in an amount of time appropriate for use in a real-time planning module?
To answer those questions, such an agent is implemented and different approaches to quantifying scene similarity are compared with respect to performance and quality of solutions. Subsequently, qualitative representations are selected from scientific literature and evaluated.


The next section describes the background of the work, followed by details of the implementation and evaluation results.
