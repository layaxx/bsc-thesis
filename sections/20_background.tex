\section{Background}\label{sec:background}

\subsection{Case-Based Reasoning}
Case-Bases Reasoning (CBR) is a method that dates back to the 1980s. It works similarly to how humans operate: Confronted with a novel situation, humans tend to think of similar situations they encountered in the past and try to adapt their previous solution to handle the new scene. % TODO: citation needed

One key advantage of CBR over approaches like Deep Learning is that not a lot of data or time needs to be spent on training. Since the agent generates cases from experiences, general knowledge of the the problem domain is not required.\cite{CBR-issues-variations-approaches}

Recently, CBR has been used in medical fields\cite{medical1, mediacl2} and robots\cite{QCBR}. % TODO: citation needed

\subsection{Qualitative Representations and Reasoning}
Unlike quantitative representations, qualitative representations try to abstract from actual values, thus increasing generalizability.


% What is CBR?
% Current Research Status of CBR?
% Why quantitative? -> abstraction from values
\subsection{Previous Work}
In previous work, we devised a strategy to use case-based reasoning to allow the agent to remember good shots und use them again in similar scenarios, even though the strategy that originally produced this shot may not have a high priority.
Case-based reasoning is a strategy in artificial intelligence that aims to modify known solutions to problems in order to make them applicable to the problem at hand.
To do so, relevant information about the problem and the associated solution need to be stored and similarity between problems must be judged.
If two problems are similar enough, the solution can be transformed to solve the new problem.
In out implementation, similarity between scenarios was judged quantitatively based on position and size of objects.


\subsection{Goals \& Approaches}

Now the goal is to devise a qualitative solution.
Instead of absolute positions, relations between objects shall be considered when determining applicability of cases.
This could lead to an improvement in performance compared to the quantitative approach, as that came with limitations as to what objects were considered and when two scenes where estimated to be transferable. % TODO: should evaluation also be run against the quantitative approach?
For example, a few pixels could make the difference between an object being irrelevant or an actual obstacle.
But due to thresholds we used in the quantitative approach, this distinction may not occur. % TODO: add image of a level where this could be an issue
% TODO: Check wether this level could then be beaten using QualCBR

The main objective will consist of comparing subsets of quantitative relations to find those that maximize the score the agent can achieve in-game. % TODO: what relations are available?
% -> above, below, toTheLeft, toTheRight, touches, ...
% -> how should empty space be represented (with predicates, as separate object, ...)


