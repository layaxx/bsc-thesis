What does a case in the context of AngryBirds consist of?
Cases usually describe a specific problem.
A typical problem in AngryBirds is a target that needs to be destroyed but may be sheltered by obstacles.
The problem can therefore be described as a set of objects that are in spatial relation to each other.

The next component is a suggested solution, which promises to solve the given problem.
In AngryBirds, such problems are solved by performing a shot with a bird at an object. Each case is therefore associated with a description of a shot, which can be executed to solve the problem, i.e. destroy the target.
Additionally, each case contains a description of the effects which the shot is expected to have on the objects, i.e. which of the objects will be destroyed and which will move.

\paragraph{The situation}
A description of the situation is necessary for the planner to determine whether a given scenario is sufficiently similar for the case to provide meaningful input.
For the qualitative \ac{CBR}, this description consists of every object that is relevant to a problem and a set of relations, that provide more information about the qualities of those objects and their relative positions.

Because it is non-trivial to find out which objects are relevant to a problem, a heuristic is used here: an object is deemed relevant if the shot either moves it or destroys it.


\paragraph{The shot}
For the quantitative \ac{CBR}, described in \ref{subsec:previous}, we used the target coordinates of the original shot, which were then transformed to match the new scenario.
Because this transformation of coordinates will not happen during the qualitative CBR process, a shot is instead represented by the object the original shot had targeted.
A drawback of this approach is the inability to represent complex shots not directly targeting one object.
Examples of such complex shots would be rebound shots.

Additional information is added, as along with the target object, qualitative predicates about the shot are saved. 
These consist of a classification of how and where the target was (supposed to be) hit, i.e. the impact angle and whether the exact target point was above, below or at the center of the target object.
Both are only estimations, however. Due to technical restrictions, we can record where the agent was trying to hit the object, but not where the shot did hit.

The impact angle of a shot is a value in $(-90;90)$. A shot that hits the target straight on has an impact angle of 0, while an angle of plus/minus 90 degrees would correspond to a shot hitting the target directly from below/above respectively. Shots are categorized based on the absolute value of the impact angle: If it is less than 20 degrees, the shot is called ``low'', if it is less than 40 degrees, ``medium'', otherwise, the shot is classified as ``high''.

\asfigure{fig:classified-shots}{data/situation8-1.pdf}{Shots generated by the planner. Shots are classified, red corresponds to high, gray to medium and blue to low shot.}{15}


\paragraph{The Effects}
The effects describe how the original shot interacted with objects in the scene. 
An object can either be destroyed, moved or not affected at all by a shot. There are two main use cases for the expected effects a shot might have.
The first is in the planning phase, where these might be used to determine whether this shot will improve the situation at hand, thus enabling complex planning processes for multiple shots in a row.

After executing a shot, the expected effects can be compared to the observed effects, checking for every object that should have been affected whether and how it was affected.
A complete match means every object that should have been destroyed was destroyed and every object that should have moved has moved.
Depending on the circumstances, a case can still be considered a success even though not every object was affected as expected.
For example, if an object should have been moved but was actually destroyed, it is not a perfect match anymore, but depending on how the effects are used in the planning stage, this might or might not influence whether the shot still achieved the goal. Section \ref{subsec:experimental-predicates} discusses under what conditions a case application is considered successful for the evaluation.

If the effects match closely even though the situation before the shot was not a perfect match, this can be used to update the case in order to loosen restrictions on the situation. 
For example, if a case with large wooden objects has the same effects when the wooden objects are small, the restriction on size could either be removed entirely or could be loosened to either small or big, leaving medium objects out until they have been observed to work as well.
If the match was not perfect and the observed effects were different from expectations, a new case can be generated.

If, on the other hand, the situation before the shot did match well but the observed effects were notably different from expectation, the case restrictions could either be made more strict or the entire case could be discarded.

Another option for learning would be to associate cases with a confidence value which is increased once a case is successfully applied and decreased on failure.

\paragraph{When should a new case be added?}
This question goes back to an entirely different, not yet solved problem: Determining how good a shot was.
This is nontrivial, as a shot might not destroy a target but still improve the overall situation by removing obstacles and enabling better shots at remaining targets.
A shot that takes out a target might also not be very good if it leads to other targets becoming obstructed and harder to hit.
Because classifying the quality of a shot is out of scope for this paper, we assume that shots are good if they destroy at least one target.

When the agent has executed a shot that meets this criterion, one of two things can happen: The shot was already based on a \ac{CBR} shot, then the underlying case can be updated based on the observed effects. Otherwise, a new case is generated using this shot and its effects.

To avoid trivial shots, such as direct hits at a pig, which might only affect this one object and thus match every situation, shots are only added to the database if the target object is not a pig.

\begin{lstlisting}[caption=exemplary case information outside of Prolog]
{
  "targetID": "ice8",
  "birdType": "RedBird",
  "effects": {
    "movedIDs": ["ice28"],
    "destroyedIDs": ["ice7", "pig1", "ice8"]
  },
  "shotType": "MEDIUM",
  "optionalPredicates": [
    { "predicateName": "hasForm", "args": ["Ice8", "bar"] },
    //...
  ],
  "caseIndex": 5
}    
\end{lstlisting}

\begin{lstlisting}[label=lst:case-prolog, language=Prolog, caption=exemplary case information in Prolog]
case(cbr_5, [medium, center]).
testAssignment(cbr_5,[Ice8, Ice28, Ice7, Pig1],Conflicts, Score):-
    hittable(_,Ice8),
    find_conflicts([before(Ice28,Pig1,yAxis),hasForm(Ice7,block), //...
        ], Conflicts),
    length(Conflicts, CL),
    Score is 1 -(CL / 24).
randomAssignment(cbr_5, [Ice8, Ice28, Ice7, Pig1]):-
    findall(ID,object(ID,rect,ice),Icerect),
    random_permutation(Icerect, TargetIcerect),
    member(Ice8, TargetIcerect),
    hittable(_, Ice8),
    delete(TargetIcerect,Ice8, WithoutTarget),
    [Ice28,Ice7|_] = WithoutTarget,
    findall(ID,object(ID,ball,pork),Porkball),
    random_permutation([Pig1|_],Porkball),
true.
\end{lstlisting}