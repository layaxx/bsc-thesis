Apart from the rather obvious properties of single objects, such as their general shape("rectangle", "circle", or "poly"), their material("ice", "wood", "stone", "tnt") and their rough dimensions, a key constraint for similarity of scenes is the relative position of objects to each other.

There are different approaches with different levels of precision. Four different ways are implemented and compared, see \ref{subsec:experimental-predicates}.

The first three sets of predicates are similar to each other:
The most precise description is achieved using the Extended Interval Algebra introduced by Renz \cite{Renz-ERA}. Because it was intended for determining stability of structures, this provides very granular information about objects that are touching or overlapping.
Whether this granularity is even useful in an environment where the computer vision introduces uncertainty will be shown in the evaluation section.

Similar but reduced sets of predicates are the Interval Algebra (IA) and Reduced Interval Algebra (RIA). While not technically subsets of EIA, they group several of the relations available in EIA.

To evaluate which set of predicates performs best, again a database of cases is generated by running the agent in create-only mode.
For every case, enough information is saved such that predicates can be generated after the agent has finished. Effectively, this means that the the target object and the pre- and post-shot screenshots are saved. For each set of predicates, the corresponding Prolog rules can now be generated by loading the images, generating the cases as the agent would during a normal run and writing the prolog rules to the database.

For each set the agent can then be run in CBR-only mode. This means that the agent tries to find a matching CBR case from the database and executes the best available. If no case matches better then a threshold, the agent targets a pig directly. In this mode, no new cases are generated and cases are not updated, to limit randomness. Due to stochastic elements in the scene matching, which have been discussed earlier, randomness cannot be entirely avoided.

Because the agent runs on the same database for every set of predicates, the results are comparable. For each set of predicates, the appropriate case database was generated and the agent was run for 60 minutes. The Level selection module was set to first iteratively play every level before any levels can be attempted again., further improving comparability between runs.
Applicability of cases was determined with 50 iterations of the \ac{WMCH} algorithm.

During runs, every action performed ny the agent is logged, so that the run can be analyzed afterwards. Relevant data includes the strategies that were considered and which strategy was chosen for each shot, the results of a shot, expected and actual effects for CBR shots and the score if a level is beaten.

This information allows the run to be reconstructed and analyzed at any time.

\asfigure{fig:run-analysis}{data/run_analysis}{Screenshot from the Analysis tool, showing a filtered chronology of a run. Level 2 Shot 2 shows both considered and choosen strategies}{15}

For this purpose, the quality is judged by the maximum score the agent achieves and by whether a case kills the pigs it promised to, regardless of other objects that should have been affected.

\ref{fig:eval-strat-frequency} shows that all versions use cases from \ac{CBR} for about two-thirds of shots. As expected, the least constraining set of relations, \ac{RIA}, leads to more matching cases. This strategy also managed to beat the highest number of levels, although the difference between versions is not too great, see \ref{fig:eval-number-of-cleared}. Notably, it was the only agent to beat Level 12, see \ref{fig:eval-cleared}.

With the exception of \ac{IA}, the derivatives of Interval Algebras where the before predicate is split based on whether objects are close enough to hit each other, actually performed worse than the base versions by score and number of levels cleared. This is a somewhat unexpected result and, in combination with the good performance of \ac{RIA}, suggests that the \ac{CBR} strategy benefits more from a higher level of abstraction.

Different sets of relations mostly lead to the same case being the best fit for a given scenario. This analysis can only be done for the first shot of any given level, because only then the situation is known to be equal between runs. This can be seen in \ref{fig:eval-best-case}. The log from the case generation shows that for example in Level 4, two different perfect matches are available. both of which where found by different agents.

\asfigure{fig:eval-all}{data/eval_all}{Maximum Score by strategy on each level}{15}
\asfigure{fig:eval-cleared}{data/eval_cleared}{cleared levels for each strategy}{15}
\asfigure{fig:eval-number-of-cleared}{data/eval_number_of_cleared}{number of cleared levels by strategy}{15}
\asfigure{fig:eval-scores-no-zero}{data/eval_scores_no_zero}{Maximum Score on cleared levels}{15}
\asfigure{fig:eval-sum-of-scores}{data/eval_sum_of_scores}{Sum of scores over every level}{15}
\asfigure{fig:eval-strat-frequency}{data/eval_strat_frequency}{Frequency of shots that were considered and executed}{15}
\asfigure{fig:eval-success}{data/eval_success}{Frequency of \ac{CBR} shots destroying some or all pigs they were expected to}{15}
\asfigure{fig:eval-best-case}{data/eval_best_case}{Cases found to be best fit for the first shot in each level}{15}

Regardless of which relations are available, every pair of objects is described by exactly two relations, one for each axis. Some of these are redundant due to transitive properties.
Consider a one dimensional example with objects a, b and c: if a is left of b and b is left of c, a must be left of c, thus it is not technically necessary to save the a left of c relation.
To make matching subsets of scenes easier, all relations are kept even if they are redundant. A subset of scenes in this case would be objects a and c, but not b.

In contrast, inverse relations are not considered. If object a is before object b, then b is after a. Saving both would again be redundant, and because it does not simplify subset matching, inverses are discarded.

One thing to keep in mind is that with a growing number of objects in a case, the importance of size and material predicates decrease relative to the positional predicates. Adding an object to a case that already has $n$ objects adds one size and one material predicate but $2*n$ positional predicates. This could be mitigated by adding weights to the predicates.

While the Interval Algebras are very precise when objects touch or overlap, they provide little distinction for objects that are farther apart. It seems reasonable to differentiate between whether object A is close enough to object B such that A could impact B when falling over. Both situations would be encoded as before by RIA/IA/EIA either way, but are expected to have impact on the transferability of cases.
A possible solution would be splitting the before predicate for those algebras into more granular relations.

Another, fundamentally different way predicates could work is the Elevated Oriented Point Algebra (EOPRA), proposed by \cite{EOPRA-Perico2016CollaborativeCO}. Instead of relations for x and y axis, position between two objects is encoded as (relative, qualitive) distance and (qualitive) angle.
