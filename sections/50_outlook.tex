\section{Summary and Outlook}\label{sec:outlook}
This paper documents the implementation of a quantitative \ac{CBR} approach to beat AngryBirds. Different methods for comparing similarity between and describing scenes have been compared, showing a (slight) improvement over baseline performance.

Possible future improvements could be made in the scene matching algorithm, which does not always find an optimal solution.
One addition could be a similarity metric, which takes into account how much constraints are violated, instead of the binary approach presented here.
\cite{QCBR} uses such a metric for \ac{EOPRA}. For example, if two objects should have the relation (close, top-left) but instead have (medium, top-left), this would be a better match than (far, right). With the presented approach, both are counted as violation with equal weight.

One of the promises of the \ac{CBR} approach is the prediction of effects on a scene it provides, which could in the future be integrated into a more complex planning system that uses this information to plan ahead several shots.
This would require a more complex assessment of shots as well, such that shots may be included that do not directly kill targets but improve the layout of the level for future shots. Such a scenario could also profit

Because \ac{CBR} does not require complete knowledge, due to learning from its own experiences, it might proof helpful in the Novelty Track of the AIBirds Competition, where objects with unknown properties are part of the challenge, thus decreasing the effectiveness of more rigid strategies like targeting TNT.
